\documentclass[Japanese]{dicomopapers}
%\documentclass[Japanese,noauthor]{dicomopapers}
\usepackage[dvipdfmx]{graphicx}
\usepackage{latexsym}
\usepackage{url}
\renewcommand{\baselinestretch}{1.0}

\def\Underline{\setbox0\hbox\bgroup\let\\\endUnderline}
\def\endUnderline{\vphantom{y}\egroup\smash{\underline{\box0}}\\}
\def\|{\verb|}



\begin{document}

% 和文表題
\title{BLEビーコンを用いた物体状態推定手法}
% 英文表題
\etitle{Proposal of Object State Estimation Method \\ using BLE Beacon}

% 所属ラベルの定義
\affiliate{1}{愛知工業大学 情報科学部情報科学科}

\author{大鐘 勇輝}{YUKI OGANE}{1}
\author{水野 涼雅}{RYOGA MIZUNO}{1}
\author{梶 克彦}{KATSUHIKO KAJI}{1}




\begin{abstract}
近年IoTの普及によって様々なセンサが家電に取り付けられ, そこから得られる情報によりライフログデータの取得が可能となった.
ライフログデータはセンサが搭載されている家電では収集が容易であるが, 特殊なセンサの無い家電や扉, 椅子といった家具ではデータの収集が難しい.
これまでそのような家電や家具などの状態推定には加速度センサやWi-Fiの電波が用いられてきたが, 移動の有無や推定対象物の大きさによって推定精度が大きく左右される問題があった.
そこで本研究では, Bluetooth Low Energyビーコン(以下BLEビーコンと呼称)を家電や家具などモノの中に直接入れ, 状態によって変化するBLEビーコンの電波強度をもとに推定を行う手法を提案する.
BLEビーコンの電波は微弱であるため, 環境の変化による電波の乱れが発生しやすい.
そのため提案手法では, 取得した電波強度データに対しデジタルフィルタの一つであるローパスフィルタを適用しノイズの除去を行う.
また, 本手法では推定対象物の移動も考慮するため, 簡単な閾値処理だけでは推定が困難である.
そこで, 安定センシング区間という概念を導入し, 電波強度が安定している区間を見つけ状態を判断する閾値を動的に変えることによって推定精度の向上を試みる.
安定センシング区間の判定には, 吉澤ら\cite{ips-chube}が研究内で使用しているεチューブを利用する.

\textgt{\\キーワード} : ビーコン電波による物体の状態推定, 安定センシング区間, εチューブ, ライフログ


\end{abstract}

% 表題などの出力
\maketitle

% 本文はここから始まる
\section{はじめに}
通信インフラの整備とIoTの発展により, 現在では様々なモノがインターネットに繋がるようになった.
エアコンや照明, 家の鍵から車に至るまで, 数年前ではインターネットとは無関係だったものが今では当たり前のようにインターネットに接続されている.
IoTに対応した家電は遠隔からの機器の操作や 機器の状態の通知, 使用ログの保存といったことが可能であり, 例えばIoTに対応したエアコンでは家の外から電源のON・OFFや電気使用量のログの確認ができる.
そうした利便性から日々新たなIoT家電が生み出され続けており, 総務省が公開している資料\cite{soumusyo}によると2020年には約300億のIoTデバイスが稼働していると予想されている.
IoTデバイスの増加により家中にそれらが設置されると, そこから得られる使用データから行動パターンといったライフログのデータが取得できる.
ライフログのデータは, 環境に合わせた電力制御や老人の異常行動の検知など幅広い分野への応用が期待できる.

しかし, これらのデータはIoTに対応した家電でないと取得できず, 金庫や椅子, 扉といった家電以外のモノではセンサが無いためデータ収集が不可能という問題がある.
この解決策として, (1)加速度センサや回転センサを使う方法と, (2)Wi-Fi電波のチャネルの状態情報を使った方法がある.
(1)の方法は直接モノの動きを観測できるため安定した精度で状態推定ができる反面, センサー単体での動作が難しく取得したデータの保存・解析に専用の機器が必要である. %消費電力が大きく定期的に電池の交換が必要であるでも良いかも
また(2)の方法では対象物のそれぞれにセンサを付ける必要が無い一方, 状態の推定にWi-Fi電波の反射波を用いるという特徴から小さな対象物への適用は不得意である.

そこで本研究では汎用的な機器のみで動作でき, かつ小さな対象物でも高い精度で状態推定できる手法を提案する.
具体的には日常の生活空間内における家電・家具の内部にBluetooth Low Energyビーコン(以下BLEビーコン)を設置し, 送信される電波をスマートフォンで受信した際の受信電波強度をもとにモノの状態推定を行う手法である.
BLEビーコンは小型・省電力なため長期間の稼働が可能であり, また電波が微弱なので障害物の有無により電波強度が変化しやすく状態推定に適している.
加えてBLEビーコンの電波はスマートフォンで受信できるため, 電波の受信に特別な機器を必要とせず, UUID, major, minorの情報から各BLEビーコンを識別することができるため, 1台のスマートフォンで複数のBLEビーコンを監視することができる.

システム概要図を図\ref{abst}に示す.
本手法では, BLEビーコンを金庫や冷蔵庫であれば開閉を行う蓋や扉の部分に, 椅子では座面のクッション下へ設置し推定を行う.
この時, 受信したデータをそのまま使用するとノイズの影響で正確に推定を行うことができない.
そのため, ノイズを取り除くため受信したデータに対してフィルタ処理を施し, 状態の推定を行う.

本稿の構成は以下の通りである.
2章では, 屋内日常物における状態推定に関する既存研究を紹介し, その問題点を述べる.
3章では, 既存手法における問題点を解決するために, 本研究のBLEビーコンを使用してモノの状態推定行う手法を述べる.
4章では, 本研究で提案した手法の評価と考察を行う.
最後に5章で, 本研究のまとめと考察を行う.


%ーーーーーーーーーーーーーーーーーーーーーーーーーーーーーーーーーーーーーーーーーーーーーーーーー
%ーーーーーーーーーーーーーーーーーーーーーーーーーーーーーーーーーーーーーーーーーーーーーーーーー

% 図の挿入
\begin{figure}[ht]
 \centering
 \includegraphics[width=8.5cm]{abst.png}
 \caption{システム概要図}
 \label{abst}
\end{figure}

\section{関連研究}
室内にある物の状態推定には, 加速度センサや振動センサなどを利用する方法やWi-Fiの電波を利用する方法など様々な手法が提案されてきた.
前川ら\cite{TagAndThink}は様々なセンサを搭載したセンサノードをモノに取り付け, そこから得られるセンサ情報と事前に用意しておいたそのモノ固有の状態遷移図を比較することにより, 自動でモノの状態と何に付けられているのか推定をしている.
角速度や照度などから, センサノードが取り付けられている状況や状態変化を検出するという手法である.
この手法ではセンサノードがどんなモノに付いていてどんな状態変化をしたかを推定することが可能である.


消費電力の変化から電気機器の状態推定を行っている研究がいくつかある.
上田ら\cite{sairyu}は機器やコンセントごとの消費電力を計測できる細粒度電力センサを使用し, 電気機器の電力消費の変化から浪費電力の検出・分類を行っている.
また, 山田ら\cite{energy}は電気機器の運転モードの切り替えや開閉などの状態変化によって起こる消費電力の変化から状態の推定を行い, 複数の電気機器の状態変化を時系列に並べることで人物の位置推定を行っている.
これら二つの手法では, 電気機器の状態変化から起こる消費電力の増減に着目し推定を行っているため, 電化製品に対しては有効な手法ある.
一方で, 電気を用いない家具や雑貨に対しては適用することが不可能である.

江田ら\cite{redLine}の赤外線で在籍推定のやつ


日常生活空間内での扉の開閉推定を行う尾原ら\cite{WifiChannel}の研究では, Wi-Fiのチャンネル状態情報を用いて物体の移動から発生するドップラー効果や電波の到来方向からドアの開閉状態を検知している.
この研究では推定の対象物は扉でありある程度大きさがあるモノであるため, Wi-Fiチャネル状態情報に影響を及ぼしやすくこの手法は有効である.

しかし, 小さな箱の開閉や移動した先での状態変化を推定する事は難しいため, 我々はBLEビーコンを用いて電波強度の変化から状態変化の推定を行うことにした.
BluetoothLowEnergyの技術は近年, 広告配信や位置推定など様々な用途に使われている.
例えばSNSサービスを提供しているLINEでは, BLEビーコンを使用して決済ができるサービスを提供している.\cite{bleUse}


BLE在室の研究
BLEビーコンの受信機との距離による減衰や,障害物による電波の減衰などによって変化する電波強度をもとに, 複数の受信機を用いて位置を推定する研究が数多く行われている\cite{IoMT},\cite{tandem},\cite{blespot},\cite{LANgate}.
これらで利用されているように, BLEの電波は微弱であるため数mの距離の変化や障害物などにより電波の減衰が起こる.
この現象を利用し我々は日常の生活空間内におけるモノの状態変化推定を行う.





%ーーーーーーーーーーーーーーーーーーーーーーーーーーーーーーーーーーーーーーーーーーーーーーーーー
%ーーーーーーーーーーーーーーーーーーーーーーーーーーーーーーーーーーーーーーーーーーーーーーーーー





\section{物体内部に配置したBLEビーコンの電波強度を用いた状態推定}
本研究では冷蔵庫や金庫, 座椅子などの内部に図\ref{beacon}のBLEビーコンを設置し, その電波強度の変化からモノの状態推定を行う.
図\ref{freezer}の冷蔵庫では扉の棚部分, 図\ref{safe}の金庫では開閉する蓋の部分, 図\ref{chair}の座椅子では人が座る座面の部分や背もたれの部分のカバー裏へBLEビーコンの設置を行う.
これにより扉の開閉や人の着座などのモノの状態変化からBLEビーコンの遮蔽状態が変化するため, 外側に設置する受信機からみたBLEビーコンの電波強度が変化する.
この変化する電波強度をもとに設置対象の状態変化の推定を行う.
% 図の挿入
\begin{figure}[ht]
    \centering
    \includegraphics[width=3cm]{ble.png}
    \caption{BLEビーコン}
    \label{beacon}
   \end{figure}
\begin{figure}[ht]
    \centering
    \includegraphics[width=7cm]{regisW2.png}
    \caption{冷蔵庫に設置したビーコン}
    \label{freezer}
\end{figure}
\begin{figure}[ht]
    \centering
    \includegraphics[width=7cm]{kinkoW.png}
    \caption{金庫に設置したビーコン}
    \label{safe}
\end{figure}
\begin{figure}[ht]
    \centering
    \includegraphics[width=8cm]{tmp.png}
    \caption{座椅子に設置したビーコン}
    \label{chair}
\end{figure}

BLEビーコンの状態変化による電波強度の変化を利用した研究に池田ら\cite{BLEpkpk}の研究がある.
この研究は複数あるBLEビーコンの中から一つを手でぱかぱかすことでそのBLEビーコンを特定出来るというものである.
モノの状態変化時にはこのぱかぱかと同じ状態になるため状態の推定が可能であると考えられる.




%ーーーーーーーーーーーーーーーーーーーーーーーーーーーーーーーーーーーーーーーーーーーーーーーーー
%ーーーーーーーーーーーーーーーーーーーーーーーーーーーーーーーーーーーーーーーーーーーーーーーーー

\subsection{電波強度のデータ収集対象及び前提}
% 図の挿入
\if0
\begin{figure}[ht]
 \centering
 \includegraphics[width=3cm]{ble.png}
 \caption{BLEビーコン}
 \label{beacon}
\end{figure}
\fi

本研究では状態推定対象にBLEビーコンを設置するが, その際設置するBLEビーコンと状態推定対象物をそれぞれ紐づけておく.

%ビーコンは図\ref{beacon}のフォーカスシステムズ社のFCS1301を使用する.
BLEビーコンは図\ref{beacon}に示すBLEビーコンを使用する.
小さな変化も検知可能にするため電波送信強度は0dBm, 送信間隔は100msと設定した.
また, 本研究では受信機としてスマートフォンを使用するため, 専用のAndroidアプリケーション図\ref{phoneApp}を作成した.

\begin{figure}[ht]
    \centering
    \includegraphics[height=5.5cm]{application.jpg}
    \caption{スマートフォン用アプリ}
    \label{phoneApp}
\end{figure}
このアプリケsーションはBLEビーコンの受信電波強度を記録するだけではなく, リアルタイムでローパスフィルタの適用と安定センシング区間の判定及びそのネガポジ判定が可能である.

%Bluetoothのセンサ精度は端末毎に異なる.
%ビーコンのある電波強度変化を複数の端末で同時に収集したものを図\ref{multi-data}に示す.
%統一された実験環境下で各端末の精度を測定した結果, Zenfone4 > XperiaXZ2 > Pixel3?の順番となり, 最も望んだ精度の端末はZenfone4であった.
%以上の理由から本研究では受信機としてZenfone4を使用する.
%そのため本研究でのデータの収集にはASUSのZenfone44を使用した.





%ーーーーーーーーーーーーーーーーーーーーーーーーーーーーーーーーーーーーーーーーーーーーーーーーー
%ーーーーーーーーーーーーーーーーーーーーーーーーーーーーーーーーーーーーーーーーーーーーーーーーー


\subsection{BLEビーコン取り付け位置の検討}
BLEビーコンの取り付け位置により, 対象物が状態変化した際の電波強度変化の仕方が異なる.
箱型で蓋の開閉を行う金庫のようなものであれば, BLEビーコンを箱内部に設置する場合と蓋のうらに設置する場合が考えられる.
それぞれの場合での電波強度の変化を図\ref{transform-data}に示す.
これより箱型のものであれば蓋の裏にBLEビーコンを設置すると開閉した際の電波強度に大きな変化が起こる事がわかる.
BLEビーコンを箱の底に取り付けた場合では状態変化による電波強度の変化があまり見られなかった.
これは箱の形からBLEビーコンの電波に指向性が持たされてしまい, 受信機のない方へ電波が飛んでしまったためと考えられる.

% 図の挿入
\begin{figure}[t]
 \centering
 \includegraphics[width=8cm]{in-out.png}
 \caption{状態変化と電波強度の変化}
 \label{transform-data}
\end{figure}

さらに, 対象の材質により電波を通しやすいために状態変化しても電波強度に大きな変化が現れないモノもある.
そこで図\ref{adapter}のように電波に指向性をもたせるアダプタを取り付けることで開閉などの状態変化で電波強度に変化が出るようにする.


% 図の挿入
\begin{figure}[ht]
 \centering
 \includegraphics[width=8cm]{adapta_compare.png}
 \caption{指向性アダプタ有り無しの画像}
 \label{adapter}
\end{figure}

指向性アダプタを取り付けることで開閉などの状態変化による電波強度の変化だけではなく, 状態変化によって電波の向きが変わることで, 受信機でキャッチ出来る範囲が変わるため図のように電波強度に大きな変化が現れる.

状態変化により大きく変化した電波強度をもとに状態推定を行っていく.
%ーーーーーーーーーーーーーーーーーーーーーーーーーーーーーーーーーーーーーーーーーーーーーーーーー
%ーーーーーーーーーーーーーーーーーーーーーーーーーーーーーーーーーーーーーーーーーーーーーーーーー

\subsection{状態推定アルゴリズム}
本手法ではBLEビーコンから発せられる電波をスマートフォンで受信し, その値から閾値を用いて状態を推定する.
図\ref{bank-opcl}は金庫の開閉を行ったときの電波強度の値に, 移動平均を用いたローパスフィルタをかけて小さな揺らぎを除去したグラフである.

\begin{figure}[t]
 \centering
 \includegraphics[width=8cm]{lowpath_compare.png}
 \caption{ローパスフィルタ適用前後の電波強度グラフ}
 \label{bank-opcl}
\end{figure}

収集した電波強度の値を正規化し0から1の値に変換する.
一時的な障害物があった場合や状態変化に際した人間の影響などにより電波強度にローパスフィルタでは除去しきれない揺らぎが発生する事があるため, 梶ら\cite{sensing-area}が提案している安定センシング区間という概念を利用し推定精度の高精度化を図る.
安定センシング区間とは一定時間以上センシングが安定して行えている区間を指す.
本手法はデータをスマートフォンで収集した後に適用可能なオフライン手法である.


\subsubsection{安定センシング区間}
本手法では推定対象物の移動も考慮するため, 安定センシング区間の判定に吉澤らの提案手法\cite{ips-chube}内で利用されているεチューブを利用する.
物体が状態変化をせず移動した場合, 徐々に電波強度は変化するため単純な閾値を用いた推定では長距離移動した際に誤判定が起こってしまう.
εチューブは一つ前の電波強度の値と現在の電波強度の値を使用し, 現在の値が一つ前の値の上限・下限の閾値範囲内に収まっているかどうかで判定を行い, 一定時間以上閾値内に収まっていた場合にそこを安定センシング区間とする.
この時の上限・下限閾値を今回使用したBLEビーコンの特徴や推定対象に合わせ定める.
% 図の挿入
\begin{figure}[ht]
    \centering
    \includegraphics[width=8.5cm]{bokoboko.png}
    \caption{BLEビーコンをただ置いて測定した電波強度グラフ}
    \label{nomal-data}
\end{figure}

%@@@@@@@@@@@@@@@@@@@@@@@@@@@@@@@@
図\ref{nomal-data}はBLEビーコンをただ置いた状態での電波強度の値にローパスフィルタを適用したグラフであり, ここからBLEビーコンの電波は周期的に電波強度の変化が見られることがわかる.
このBLEビーコン特有の周期的な強度変化により安定センシング区間の判定が不安定になることがあるため, 適切な閾値を設定する必要がある.
対象の状態変化による電波強度の変化とBLEビーコン特有の電波強度の変化を考慮して動的に閾値を設定する.

%pythonプログラムの説明
pythonプログラムは最初にcsvファイルからBLEビーコン電波取得時間と受信電波強度を読み込む.
読み込んだ受信電波強度の数値は0〜1の値に正規化し, ローパスフィルタをかけてノイズを取り除く.
次に安定センシング区間を見つけるため受信電波強度の時系列データに対してεチューブを適用する.
εチューブは時系列に並んだデータを捜査し, 予め設定しておいた閾値内に受信電波強度の値が収まっているか確認する.
もし閾値を超えた場合は, 超える直前の受信電波強度±X(Xは任意の値)の値を新しい閾値として設定する.
これを繰り返し安定センシング区間の判定が終わったら, 最後に見つけたそれぞれの安定センシング区間がネガティブな状態かポジティブな状態か判定を行う.
判定は, 各安定センシング区間の受信電波強度の平均値が受信電波強度の中央値に0.1を足した値より大きいか小さいかで判定を行う.


%@@@@@@@@@@@@@@@@@@@@@@@@@@@@@@@@@@
% 図の挿入
\begin{figure}[ht]
 \centering
 \includegraphics[width=7.5cm]{compare.jpg}
 \caption{BLEビーコンただおきのデータ}
 \label{compare}
\end{figure}


\subsubsection{状態変化の推定}
推定結果の状態をネガティブな状態とポジティブな状態の2つの状態とする.
ポジティブな状態とはBLEビーコンが外に出ていて電波の受信がしやすく、電波強度の値が高い状態の事を指す. 具体的には金庫の開閉であれば開いてる状態である.
ネガティブな状態とはBLEビーコンが外に出ておらず電波の受信が難しく、電波強度の値が低い状態の事を指す. 具体的には金庫の開閉であれば閉じてる状態である.
判定された安定センシング区間内の平均値をそれぞれ求め,すべての安定センシング区間の平均値の中央値より高いか低いかでネガティブポジティブの判定を行う.



%ーーーーーーーーーーーーーーーーーーーーーーーーーーーーーーーーーーーーーーーーーーーーーーーーー
%ーーーーーーーーーーーーーーーーーーーーーーーーーーーーーーーーーーーーーーーーーーーーーーーーー


\section{評価実験}

本稿で提案した手法の推定精度を確かめるため, いくつかのモノへBLEビーコンを設置し評価実験を行った.
BLEビーコンを設置する対象は, 冷蔵庫, 金庫, 座椅子とし, 金庫と座椅子については移動を考慮した状態変化に対し推定精度を測定した.
また, 電波強度情報の収集には専用のスマートフォンアプリケーションを作成して使用しており, BLEビーコンを識別するためのUUID, major, minorをBLEビーコンを設置する対象と紐付けし, どのBLEビーコンの値がどの対象物の物なのか把握できるようにしている.
評価の方法は開閉などの状態変化をランダムな間隔で行い, その推定結果と正解ラベルを比べて正答率を算出して行う.

\subsection{実験端末の選定}

\begin{figure}[ht]
    \centering
    \includegraphics[width=8cm]{mix.png}
    \caption{複数端末で取ったデータ}
    \label{multi-data}
\end{figure}

Bluetoothのセンサ精度は端末毎に異なる.
そこで本研究において最適な結果を得られる端末を選定するため, 同一環境下で測定精度の比較を行った.
比較結果を図\ref{multi-data}に示す.
3機種の精度を比較した結果, XperiaXZ2では小さなスパイクノイズが多く発生し, Zenfone4とNexus6Pでは小さなスパイクノイズは少ないものの大きなスパイクノイズが低い確率で発生することを確認した.
また, Nexus6Pでは30秒, 40秒付近で電波強度の変化が緩やかになっており正しく変化を示せていないことを確認した.
本研究で提案する手法は電波強度の急激な変化と安定センシング区間の判定が鍵となってくる.
そのため, スパイクノイズが少ないこと, きちんと変化が捉えられることが重要である.
以上の理由から本研究では一番ノイズが少なく, きちんと変化を捉えられるZenfone4を受信機として使用する.
また, 本研究では金庫などの移動を伴う比較的小さなモノにもBLEビーコンを取り付けるため, 使用するBLEビーコンはできるだけ小さい方が望ましい.
そのため, 軽量・小型という理由からフォーカスシステムズ社のFCS1301\ref{beacon}を使用する.


\subsection{冷蔵庫の開閉における推定精度の測定}
図\ref{freezer}, 図\ref{refrigerator_position}のようにセンサと受信機を設置して評価実験を行った.
実験は日常の使用を模倣し, 冷蔵庫のドアを開けて, 中からペットボトルを取り出し, ドアを閉めるという動作をランダムな間隔で行い, その状態の変化を捉えることができるかで評価を行った.
推定結果を図\ref{refrigerator_graph}に, 正解率の一覧を表\ref{refrigerator_fig}に示す.
図\ref{refrigerator_graph}の白色の部分は安定センシング区間ではない状態を 緑色の部分はネガティブな状態(ドアが閉まっている状態)を 赤色の部分はポジティブな状態(ドアが開いている状態)を示している.

同様の行動を3回行った結果1回目だけ不正解が一度あり, 結果として全体の正解率は95.8%となった.
誤判定された400秒付近のグラフを見てみると他と比べて受信電波強度の変化が小さく, 安定センシング区間と判断されていないことが分かる.
このことから安定センシング区間を判定するパラメータを調整すれば誤判定を避けることが出来ると考えられる.

\begin{figure}[ht]
    \centering
    \includegraphics[height=5cm]{refrigerator_position.png}
    \caption{冷蔵庫と受信機の位置関係図}
    \label{refrigerator_position}
\end{figure}

\begin{figure}[ht]
    \centering
    \includegraphics[width=8cm]{refrigerator_graph.png}
    \caption{冷蔵庫の状態推定結果グラフ}
    \label{refrigerator_graph}
\end{figure}



%@@@@@@@@@@@@@@@@@@@@
%  ポジティブな状態だけ検出できた割合の結果 ではなくネガティブポジティブ両方の検出数から割合を出すべきでは????
%@@@@@@@@@@@@@@@@@@@@
\begin{table}[htb]
    \begin{center}
        \caption{冷蔵庫の開閉における状態推定精度}
        \label{refrigerator_fig}
        \begin{tabular}{|c|c|c|c|} \hline
        試行回数 & 正答率 & 正解数 & 不正解数 \\ \hline
        1 & 87.5% & 7 & 1 \\ \hline
        2 & 100% & 8 & 0 \\ \hline
        2 & 100% & 8 & 0 \\ \hline \hline
        累計正解率 & \multicolumn{3}{c|}{95.8%} \\ \hline
        \end{tabular}
    \end{center}
\end{table}



\subsection{金庫の開閉における推定精度の測定}
図\ref{safe}, 図\ref{kinko_position}のようにセンサと受信機を設置して評価実験を行った.
金庫は図\ref{kinko_position}に赤丸で示したように3箇所の場所に移動させて蓋の開閉を行い, 推定中にモノの移動が行われても状態推定が可能かを確かめた.
推定結果を図\ref{kinko_graph}に, 正解率の一覧を表\ref{kinko_fig}に示す.
図\ref{kinko_graph}の白色の部分は安定センシング区間ではない状態を 緑色の部分はネガティブな状態(蓋が閉まっている状態)を 赤色の部分はポジティブな状態(蓋が開いている状態)を示している.

同様の行動を3回実験を行った結果, 3回とも正しく蓋の開閉を検知することができた.
しかし, 1回目の2番の場所での測定結果を見てみると, 蓋が閉まった後も数秒間蓋が開いていると推定されてしまっている.
蓋が閉まって電波強度が弱まった160秒付近のグラフを見てみると変化終了後のグラフがほぼ真横に一直線になっている.
このため蓋の開閉が終わった後もポジティブな状態と判定されてしまい誤判定が発生したのだと考えられる.
また, このことから冷蔵庫と同様に安定センシング区間を判定する閾値などのパラメータを変更すればこのような誤判定は避けられるのではと考えられる.

\begin{figure}[ht]
    \centering
    \includegraphics[width=8cm]{kinko_position_fig.png}
    \caption{金庫と受信機の位置関係図}
    \label{kinko_position}
\end{figure}

\begin{figure}[ht]
    \centering
    \includegraphics[width=8cm]{kinko_graph.png}
    \caption{金庫の状態推定結果グラフ}
    \label{kinko_graph}
\end{figure}

\begin{table}[htb]
    \begin{center}
        \caption{金庫の開閉における状態推定精度}
        \label{kinko_fig}
        \begin{tabular}{|c|c|c|c|} \hline
        試行回数 & 正答率 & 正解数 & 不正解数 \\ \hline
        1 & 100% & 3 & 0 \\ \hline
        2 & 100% & 3 & 0 \\ \hline
        2 & 100% & 3 & 0 \\ \hline \hline
        累計正解率 & \multicolumn{3}{c|}{100%} \\ \hline
        \end{tabular}
    \end{center}
\end{table}

%ーーーーーーーーーーーーーーーーーーーーーーーーーーーーーーーーーーーーーーーーーーーーーーーーー
%ーーーーーーーーーーーーーーーーーーーーーーーーーーーーーーーーーーーーーーーーーーーーーーーーー

\section{まとめ}
本稿ではBLEビーコンの電波強度を用いて日常の生活空間内の椅子や金庫などのモノの状態推定を行う手法を提案した.
推定を行う際, 瞬間的な変化や人などの障害物の通過などにより電波の揺らぎが発生し推定結果に影響を及ぼすことが考えられるが, 安定センシング区間という概念を利用することにより高い精度で状態推定を行うことが出来る.
また, BLEビーコンに指向性を持たせるアダプタを取り付けることにより, 電波強度の変化が大きくなり推定精度を向上させることを確認した.

今後の課題として, 現状では安定センシング区間の判定に用いるパラメータは手動で設定しなければならず, 適切な値を探すためにあらかじめ何度か試行させる必要がある.
よってアルゴリズムを改良して自動で適切なパラメータを設定する手法の確率を目指す.


%ーーーーーーーーーーーーーーーーーーーーーーーーーーー
%ーーーーーーーーーーーーーーーーーーーーーーーーーーーーーーーーーーーーーーーーーーーーーーーーー



\begin{thebibliography}{20}
 \bibitem{soumusyo}『平成29年版情報通信白書』: 入手先 〈\url{http://www.soumu.go.jp/johotsusintokei/whitepaper/ja/h29/pdf/n3300000.pdf}〉, p.125, (参照 2019 年 4 月 24 日).
 \bibitem{TagAndThink}前川拓也, 柳沢豊, 岡留剛. Tag and Think:センサネットワークを前提としたモノ自身とその状態の推定, 情報処理学会研究報告ユビキタスコンピューティングシステム(UBI), 2007(14(2007-UBI-013)), pp. 211-218, 2007.
 \bibitem{sairyu}上田泰嵩, 梶克彦, 河口信夫. 細粒度電力センシングによる浪費電力の検出, マルチメディア、分散、協調とモバイル(DICOMO)シンポジウム論文集, pp. 1817-1821, 2010.
 \bibitem{energy}山田祐輔, 加藤丈和, 松山隆司. スマートタップネットワークを用いた家電の電力消費パターン解析に基づく人物行動推定, 研究報告ユビキタスコンピューティングシステム(UBI), 2011(4(2011-UBI-31)), pp. 1-6, 2011.
 \bibitem{redLine}江田政聡, 賀新剛, 中根傑, 横山昌平, 福田直樹, 峰野博史, 石川博, 赤外線センサを用いた在席推定に基づく照明制御手法の提案, 第4回データ工学と情報マネジメントに関するフォーラム(DEIM フォーラム 2012).
 \bibitem{WifiChannel}尾原和也, 前川卓也, 村上友規, アベセカラヒランタ. Wi-Fiチャネル状態情報を用いた教師無し学習によるドアの開閉検知手法, 情報処理学会研究報告ヒューマンコンピュータインタラクション(HCI), 2018(1(2018-HCI-180)), pp. 1-7, 2018.
 \bibitem{bleUse}LINE公式ブログ: 入手先 〈\url{http://official-blog.line.me/ja/archives/73098915.html}〉 (参照 2019 年 4 月 24 日).
 \bibitem{IoMT}堀川三好, 工藤大希, 岡本東, 村田嘉利. 移動するモノを対象とした Internet of Things の提案, 第80回全国大会講演論文集, 2018(1), pp. 471-472, 2018.
 \bibitem{tandem}浦野健太, 廣井慧, 梶克彦, 河口信夫. 配布型BLEタグとタンデムスキャナを用いた屋内位置推定手法, 情報処理学会論文誌, 60(1), pp. 58-75, 2019.
 \bibitem{blespot}渡邉洸, 高橋雄太, 大坪敦, 藤本まなと, 荒川豊, 安本慶一. BLEビーコンと反響音センシングによる屋内スポット推定, マルチメディア、 分散協調とモバイルシンポジウム2018論文集, 2018, pp. 620-626, 2018.
 \bibitem{LANgate}梶 克彦, 河口 信夫.無線 LAN 環境特異点に基づくゲート通過検出手法
 \bibitem{BLEpkpk}池田翔太, 梶克彦. BLEビーコンをパカパカ, マルチメディア、分散協調とモバイルシンポジウム2016論文集, 2016, pp. 899-904, 2016.
 \bibitem{sensing-area}梶克彦, 河口信夫. 安定センシング区間検出に基づく3次元歩行軌跡推定手法, 情報処理学会論文誌, Vol.57, No.1, pp.12-24, 2016.
 \bibitem{ips-chube}吉澤実, 高崎航, 大村廉. 加速度ベース行動認識におけるレスポンス時間短縮のためのパラメータ検討, 情報処理学会マルチメディア,分散協調とモバイル(DICOMO2013)シンポジウム論文集, pp. 647-654, 2013.
 \bibitem{en-door}Hsi-Yuan Tsai, Guan-Heng Chen, Huang-Chen Lee. Using low-cost, non-sensor-equipped BLE beacons to track people's movements, IPSN '17 Proceedings of the 16th ACM/IEEE International Conference on Information Processing in Sensor Networks, pp. 291-292, 2017.
 \bibitem{en-AreaUsed}Rachael Purta, Aaron Striegel. Estimating dining hall usage using bluetooth low energy beacons, UbiComp '17 Proceedings of the 2017 ACM International Joint Conference on Pervasive and Ubiquitous Computing and Proceedings of the 2017 ACM International Symposium on Wearable Computers, pp. 518-523, 2017.
\end{thebibliography}


\end{document}